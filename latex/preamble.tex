%!TEX program = xelatex
%!TEX options = -shell-escape
%!TEX root = nonabelions.tex

\documentclass[a4paper,oneside]{book}



%% Packages order is important
\usepackage{unicode-latex} % https://github.com/ViktorQvarfordt/unicode-latex/blob/master/unicode-latex.sty
%\usepackage[left=3.175cm, marginparwidth=2.825cm, marginparsep=3mm]{geometry}
\usepackage{tikz}
\usepackage{braids}
\usepackage{xifthen}
\usepackage{datetime}
\usepackage{graphicx}
\usepackage{lastpage}
\usepackage[section]{placeins}
\usepackage[hidelinks,unicode=true]{hyperref}
\usepackage[capitalise]{cleveref}

% For unicode font in verbatim, this is the only think requiring use of xelatex in this document.
\usepackage{fontspec}
\setmonofont[Scale=MatchLowercase]{DejaVu Sans Mono}
\usepackage{minted}
\usemintedstyle{mathematica}


\usepackage{amssymb}
\usepackage{mathbbol}
\DeclareSymbolFontAlphabet{\mathbb}{AMSb}
\DeclareSymbolFontAlphabet{\mathbbl}{bbold}


% Check unused references
% \usepackage{refcheck}

% \usepackage{showkeys}
% \setlength{\marginparwidth}{3.5cm}
% Enable showkeys for \cref and \Cref
% tex.stackexchange.com/questions/139698/usiong-showkes-and-cleveref-together
% \makeatletter
%   \SK@def\cref#1{\SK@\SK@@ref{#1}\SK@cref{#1}}%
%   \SK@def\Cref#1{\SK@\SK@@ref{#1}\SK@Cref{#1}}%
% \makeatother
% \renewcommand*\showkeyslabelformat[1]{%
%   \parbox[t]{\marginparwidth}{\raggedright\normalfont\footnotesize\ttfamily#1}}

% \usepackage{showlabels}
% \showlabels{cite}
% \showlabels{eqref}
% \showlabels{cref}
% \showlabels{bibitem}
% \renewcommand*\showlabelsetlabel[1]{%
%   \parbox[t]{\marginparwidth}{\raggedright\normalfont\footnotesize\ttfamily#1}}


%% Theorems

\usepackage{amsthm}
\usepackage{thmtools}

\declaretheorem[style=plain,numberwithin=section]{theorem}
\declaretheorem[style=plain,numbered=no,name=Theorem]{theorem*}
\declaretheorem[style=plain,sibling=theorem]{proposition}
\declaretheorem[style=plain,sibling=theorem]{lemma}
\declaretheorem[style=plain,sibling=theorem]{corollary}

\declaretheorem[style=definition,numberwithin=section]{definition}
\declaretheorem[style=definition,qed=$\diamondsuit$,numberwithin=section]{example}
\declaretheorem[style=definition,qed=$\triangle$,numberwithin=section]{remark}


% \theoremstyle{plain}
% \newtheorem{theorem}{Theorem}[section]
% \newtheorem{proposition}[theorem]{Proposition}
% \newtheorem{lemma}[theorem]{Lemma}
% \newtheorem{corollary}[theorem]{Corollary}

% \theoremstyle{definition}
% \newtheorem{definition}{Definition}[section]
% \newtheorem{example}{Example}[section]
% \AtEndEnvironment{example}{\hspace*{0pt}\hfill\ensuremath{\diamondsuit}}

% \theoremstyle{remark}
% \newtheorem{remark}{Remark}[section]


%% Macros

% \newcommand{\bbDelta}{{\mbox{$\Delta$}\hspace{-8.0pt}\scalebox{0.8}{$\Delta$}}}
% \def\bbDelta{{%
% \setbox0\hbox{$\Delta$}%
% % \rlap{\hbox to \wd0{\hss\scalebox{0.8}{$\Delta$}\hss}}\box0
% % \rlap{\hbox to \wd0{\scalebox{0.8}{$\Delta$}\hss}}\box0
% \rlap{\hbox to \wd0{\hss\scalebox{0.82}{$\Delta$}}}\box0
% }}
\newcommand{\bbDelta}{\mathbbl{\Delta}}

\DeclarePairedDelimiter\abs{\lvert}{\rvert}
\DeclarePairedDelimiter\norm{\lVert}{\rVert}
\DeclarePairedDelimiter\bra{\langle}{\rvert}
\DeclarePairedDelimiter\ket{\lvert}{\rangle}
\DeclarePairedDelimiterX\braket[2]{\langle}{\rangle}{#1 \delimsize\vert #2}

\DeclareMathOperator{\Fib}{Fib}

\newcommand*\centermathcell[1]{\omit\hfil$\displaystyle#1$\hfil\ignorespaces}
\newcommand{\centermath}[1]{\par\centerline{\begin{minipage}{\linewidth}#1\end{minipage}}\par}


\usepackage{newunicodechar}
\newunicodechar{𝟙}{\ifmmode%
\mathbbl{1}\else%
𝟙\fi}




% BEGIN FUSION STATE DIAGRAMS

\makeatletter
\newcounter{mycounter}
\newcommand{\fs}[3][]{
  \begin{tikzpicture}[scale=0.3,font=\footnotesize,anchor=mid,baseline={([yshift=-.5ex]current bounding box.center)}]
    \def\height{1.5}
    \def\offset{0.5}
    \ifthenelse{\equal{#1}{}}{\def\height{1}}{} % Smaller height if no braiding
    \setcounter{mycounter}{0}
    \@for\el:=#2\do{
      \ifthenelse{\equal{#1}{\value{mycounter}}}{
        \braid at (\value{mycounter},\height) s_1^{-1};
        \stepcounter{mycounter}
      }{
        \stepcounter{mycounter}
        \ifthenelse{\equal{#1}{\value{mycounter}}}{
        }{
          \draw (\value{mycounter}, \height) to (\value{mycounter}, 0);
        }
      }
      \node at (\value{mycounter}, \height+\offset) {$\el$};
    }
    \draw (0, 0) to (\value{mycounter}+1, 0);
    \setcounter{mycounter}{0}
    \@for\el:=#3\do{
      \node at (\value{mycounter}+0.5, -0.6) {$\el$};
      \stepcounter{mycounter}
    }
  \end{tikzpicture}
}
\makeatother

\makeatletter
\newcommand{\fswide}[3][]{
  \begin{tikzpicture}[scale=0.3,font=\footnotesize,anchor=mid,baseline={([yshift=-.5ex]current bounding box.center)}]
    \def\height{1.75}
    \def\offset{0.5}
    \def\widthfactor{2}
    \ifthenelse{\equal{#1}{}}{\def\height{1}}{} % Smaller height if no braiding
    \setcounter{mycounter}{0}
    \@for\el:=#2\do{
      \ifthenelse{\equal{#1}{\value{mycounter}}}{
        \braid[width=\widthfactor cm, height=1.25 cm] at (\widthfactor*\value{mycounter},\height) s_1^{-1};
        \stepcounter{mycounter}
      }{
        \stepcounter{mycounter}
        \ifthenelse{\equal{#1}{\value{mycounter}}}{
        }{
          \draw (\widthfactor*\value{mycounter}, \height) to (\widthfactor*\value{mycounter}, 0);
        }
      }
      \node at (\widthfactor*\value{mycounter}, \height+\offset) {$\el$};
    }
    \draw (0, 0) to (\widthfactor*\value{mycounter}+\widthfactor*1, 0);
    \setcounter{mycounter}{0}
    \@for\el:=#3\do{
      \node at (\widthfactor*\value{mycounter} + \widthfactor*0.5, -0.6) {$\el$};
      \stepcounter{mycounter}
    }
  \end{tikzpicture}
}
\makeatother

\makeatletter
\newcommand{\fswideflex}[4][]{
  \begin{tikzpicture}[scale=0.3,font=\footnotesize,anchor=mid,baseline={([yshift=-.5ex]current bounding box.center)}]
    \def\height{1.75}
    \def\offset{0.5}
    \def\widthfactor{#4}
    \ifthenelse{\equal{#1}{}}{\def\height{1}}{} % Smaller height if no braiding
    \setcounter{mycounter}{0}
    \@for\el:=#2\do{
      \ifthenelse{\equal{#1}{\value{mycounter}}}{
        \braid[width=\widthfactor cm, height=1.25 cm] at (\widthfactor*\value{mycounter},\height) s_1^{-1};
        \stepcounter{mycounter}
      }{
        \stepcounter{mycounter}
        \ifthenelse{\equal{#1}{\value{mycounter}}}{
        }{
          \draw (\widthfactor*\value{mycounter}, \height) to (\widthfactor*\value{mycounter}, 0);
        }
      }
      \node at (\widthfactor*\value{mycounter}, \height+\offset) {$\el$};
    }
    \draw (0, 0) to (\widthfactor*\value{mycounter}+\widthfactor*1, 0);
    \setcounter{mycounter}{0}
    \@for\el:=#3\do{
      \node at (\widthfactor*\value{mycounter} + \widthfactor*0.5, -0.6) {$\el$};
      \stepcounter{mycounter}
    }
  \end{tikzpicture}
}
\makeatother

\makeatletter
\newcommand{\fswider}[3][]{
  \begin{tikzpicture}[scale=0.3,font=\footnotesize,anchor=mid,baseline={([yshift=-.5ex]current bounding box.center)}]
    \def\height{1.75}
    \def\offset{0.5}
    \def\widthfactor{3}
    \ifthenelse{\equal{#1}{}}{\def\height{1}}{} % Smaller height if no braiding
    \setcounter{mycounter}{0}
    \@for\el:=#2\do{
      \ifthenelse{\equal{#1}{\value{mycounter}}}{
        \braid[width=\widthfactor cm, height=1.25 cm] at (\widthfactor*\value{mycounter},\height) s_1^{-1};
        \stepcounter{mycounter}
      }{
        \stepcounter{mycounter}
        \ifthenelse{\equal{#1}{\value{mycounter}}}{
        }{
          \draw (\widthfactor*\value{mycounter}, \height) to (\widthfactor*\value{mycounter}, 0);
        }
      }
      \node at (\widthfactor*\value{mycounter}, \height+\offset) {$\el$};
    }
    \draw (0, 0) to (\widthfactor*\value{mycounter}+\widthfactor*1, 0);
    \setcounter{mycounter}{0}
    \@for\el:=#3\do{
      \node at (\widthfactor*\value{mycounter} + \widthfactor*0.5, -0.6) {$\el$};
      \stepcounter{mycounter}
    }
  \end{tikzpicture}
}
\makeatother

\newcommand{\fsfused}[5]{
  \begin{tikzpicture}[scale=0.3,font=\footnotesize,anchor=mid,baseline={([yshift=-.5ex]current bounding box.center)}]
    \def\height{1.75}
    \def\offset{0.5}
    \node at (1, \height+\offset) {$#2$};
    \node at (2, \height+\offset) {$#3$};
    \draw (0.5, 0) to (2.5, 0);
    \draw (1, \height) to [bend left=-30] (1.5, 1);
    \draw (2, \height) to [bend left=30] (1.5, 1);
    \draw (1.5, 1) to (1.5, 0);
    \node at (0.75, -0.6) {$#1$};
    \node at (2.25, -0.6) {$#4$};
    \node at (2, 0.7) {$#5$};
  \end{tikzpicture}
}

\newcommand{\fsfusedUp}[9]{
  \begin{tikzpicture}[scale=0.3,font=\footnotesize,anchor=mid,baseline={([yshift=-.5ex]current bounding box.center)}]
    \node at (0, 1.5+0.5) {$#1$};
    \node at (1, 2.5+0.5) {$#2$};
    \node at (2, 2.5+0.5) {$#3$};
    \node at (3, 1.5+0.5) {$#4$};
    \draw (-1, 0) to (4, 0); % horizontal
    \draw (1, 2.5) to [bend left=-30] (1.5, 1.5);
    \draw (2, 2.5) to [bend left=30]  (1.5, 1.5);
    \draw (1.5, 1.5) to (1.5, 0); % vertical under bend
    \draw (0,   1.5) to (0,   0); % vertical non-bend
    \draw (3,   1.5) to (3,   0); % vertical non-bend
    \node at (-0.75, -0.6) {$#5$};
    \node at (0.75, -0.6) {$#6$};
    \node at (2, 0.7) {$#7$};
    \node at (2.25, -0.6) {$#8$};
    \node at (3.75, -0.6) {$#9$};
  \end{tikzpicture}
}

\newcommand{\fsfuseddouble}[9]{
  \begin{tikzpicture}[scale=0.3,font=\footnotesize,anchor=mid,baseline={([yshift=-.5ex]current bounding box.center)}]
    \def\height{1.75}
    \def\offset{0.5}
    \node at (1, \height+\offset) {$#2$};
    \node at (2, \height+\offset) {$#3$};
    \node at (3, \height+\offset) {$#6$};
    \node at (4, \height+\offset) {$#7$};
    \draw (0.5, 0) to (4.5, 0);
    \draw (1, \height) to [bend left=-30] (1.5, 1);
    \draw (2, \height) to [bend left=30]  (1.5, 1);
    \draw (3, \height) to [bend left=-30] (3.5, 1);
    \draw (4, \height) to [bend left=30]  (3.5, 1);
    \draw (1.5, 1) to (1.5, 0);
    \draw (3.5, 1) to (3.5, 0);
    \node at (2,     0.5) {$#4$};
    \node at (0.75, -0.6) {$#1$};
    \node at (2.5,    -0.6) {$#5$};
    \node at (4,     0.5) {$#8$};
    \node at (4.25, -0.6) {$#9$};
  \end{tikzpicture}
}

\newcommand{\fsfusedbraided}[5]{
  \begin{tikzpicture}[scale=0.3,font=\footnotesize,anchor=mid,baseline={([yshift=-.5ex]current bounding box.center)}]
    \def\height{3}
    \def\offset{0.5}
    \node at (1, \height+\offset) {$#2$};
    \node at (2, \height+\offset) {$#3$};
    \braid at (1, \height) s_1^{-1};
    \draw (0.5, 0) to (2.5, 0);
    \draw (1, 1.5) to [bend left=-30] (1.5, 1);
    \draw (2, 1.5) to [bend left=30] (1.5, 1);
    \draw (1.5, 1) to (1.5, 0);
    \node at (0.5, -0.6) {$#1$};
    \node at (2.5, -0.6) {$#4$};
    \node at (2, 0.7) {$#5$};
  \end{tikzpicture}
}

% END FUSION STATE DIAGRAM
