%!TEX root = nonabelions.tex

\chapter*{Abstract}

Anyons arise as indistinguishable quantum mechanical particles in $2+1$-dimensional spacetime as consequence of the fundamental group $π₁(𝒞ₙ)$ of the configuration space $𝒞ₙ$ of $n$ identical particles being the braid group $Bₙ$. In $d≥3$ spatial dimensions $π₁(𝒞ₙ) = Sₙ$ and the only types of identical particles are bosons and fermions. Unitary representations of $π₁(𝒞ₙ)$ determine the particle statistics which in turn gives rise to an effective repulsion, known as statistical repulsion. If the representation of $π₁(𝒞ₙ)$ is abelian the corresponding particles are known as abelian anyons and effectively give a continuous interpolation of bosons and fermions. Otherwise, the anyons are known as non-abelian and the corresponding statistical repulsing is yet largely unexplored. We employ the framework of modular tensor categories (rooted in ..) to show how the statistical repulsing of non-abelian anyons depend on the representation of $π₁(𝒞ₙ)$. In particular, the Fibonacci anyon model is studied, and as an application we show how non-abelian anyons can be used to implement topological quantum computation.

