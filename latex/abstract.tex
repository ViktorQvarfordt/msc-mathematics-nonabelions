%!TEX root = nonabelions.tex

\chapter*{Abstract}

As opposed to classical mechanics, quantum mechanical particles can be truly identical and lead to new and interesting phenomena. Identical particles can be of different types, determined by their exchange symmetry, which in turn gives rise to statistical repulsion. The exchange symmetry is given by a representation of the exchange group; the fundamental group of the configuration space of identical particles. In three dimensions the exchange group is the permutation group and there are only two types of identical particles; bosons and fermions. While any number of bosons can be at the same place or in the same state, fermions repel each other. In two dimensions the exchange group is the braid group and essentially any exchange symmetry is allowed, such particles are called anyons. Abelian anyons are described by abelian representations of the exchange group and can be seen as giving a continuous interpolation between bosons and fermions. Non-abelian anyons are much more complex and their statistical repulsion is yet largely unexplored. We use the framework of modular tensor categories to show how the statistical repulsion of non-abelian anyons depends on the exchange symmetry.  The Fibonacci anyon model is studied, for which explicit results are obtained. We also show how Fibonacci anyons can be used to implement topological quantum computation, providing topologically stable quantum information encoded in the state of non-abelian anyons that can be manipulated via the non-abelian exchange symmetry.

\newpage


\thispagestyle{plain}
\null
\begin{center}
  % Icke-Abelska Anyoner: Statistisk Repulsion och Topologisk Kvantberäkning
  {\Huge\textbf{Icke-Abelska Anyoner}}\\[1em]
  {\huge Statistisk Repulsion och\\[0.3em]Topologisk Kvantberäkning}
\end{center}
\vspace{2.5cm}
{\huge\textbf{Sammanfattning}} \\[3em]
\noindent Till skillnad mot klassisk mekanik kan kvantmekaniska partiklar vara helt identiska, vilket ger upphov till nya och intressanta fenomen. Identiska partiklar kan vara av olika slag, bestämda av deras utväxlingssymmetri, vilket i sin tur leder till statistisk repulsion. Utväxlingssymmetrin ges av en representation av utväxlingsgruppen; fundamentalgruppen av konfigurationsrummet för identiska partiklar. I tre dimensioner ges utväxlingsgruppen av permutationsgruppen och det finns bara två typer av identiska partiklar; bosoner och fermioner. Medan bosoner kan vara på samma ställe eller i samma tillstånd, repellerar fermioner varandra. I två dimensioner ges utväxlingsgruppen av flätgruppen och i princip alla typer av utväxlingssymmetrier är tillåtna, sådana partiklar kallas anyoner. Abelska anyoner beskrivs av abelska representationer av utväxlingsgruppen och ger väsentligen en kontinuerlig interpolation mellan bosoner och fermioner. Icke-abelska anyoner är mycket mer komplexa och deras statistiska repulsion är ännu mestadels outforskad. Vi använder modulära tensorkategorier för att visa hur den statistiska repulsionen för icke-abelska anyoner beror på utväxlingssymmetrin. Fibonacci anyon-modellen studeras, för vilken explicita resultat erhålls. Vi visar också hur Fibonacci anyoner kan användas för att implementera topologisk kvantberäkning, vilket ger topologiskt stabil kvantinformation som kodas i tillståndet för icke-abelska anyoner som kan manipuleras via den icke-abelska utväxlingssymmetrin.

\newpage

\chapter*{Acknowledgement}

I would like to thank my thesis advisor Douglas Lundholm at the Department of Mathematics, KTH Royal Institute of Technology. He introduced me to anyons and taught me a lot in this fascinating area of mathematics and theoretical physics. During the work of this thesis he has provided excellent guidance and pointed me in the right direction through numerous discussions.

% The door to Prof. [Last name] office was always open whenever I ran into a trouble spot or had a question about my research or writing. He/She consistently allowed this paper to be my own work, but steered me in the right the direction whenever he thought I needed it.

I would also like to thank Eddy Ardonne and Thors Hans Hansson at the Department of Physics, Stockholm University for their input and fruitful discussions. As the second reader of this thesis Eddy Ardonne has provided valuable comments and remarks. Thank you.
