%!TEX root = nonabelions.tex

\chapter{Introduction}

Classically, identical particles can always be distinguished by their position. This is not the case in quantum mechanics, identical particles can be truly indistinguishable. This effect is essentially due to non-localization; quantum mechanical particles do not have exact positions. The concept of identical particles gives rise to fundamentally new and interesting phenomena.

% Nature contains many types of particles, all of which can be classified into two distinct types; bosons and fermions.
All particles in nature can be classified into two distinct types of identical particles; bosons and fermions.
This classification is essentially determined by how a pair of identical particles behave. If nature allows the particles to be in the same quantum state we say that they are bosons, otherwise we call them fermions. This is known as particle statistics.

This is not the full story. The classification into exactly two types of identical particles is true only in three spatial dimensions. Clearly, space around us is three-dimensional (ignoring any curled up dimension), which is why we typically only observe bosons and fermions. However, it is possible to trick nature into being two-dimensional by effectively freezing motion in one of the three spatial dimensions, an example of this is the fractional quantum Hall effect.

Contrary to what one may expect, removing one spatial degree of freedom allows for the existence of more types of particles. In two spatial dimensions essentially \emph{any} type of identical particles is allowed, we call such particles \emph{anyons}, originally due to F. Wilczek \cite{wilczek}. In fact, a continuous interpolation between bosons and fermions arise, known as abelian anyons. Even more exotic particles are allowed, known as non-abelian anyons, having higher-dimensional internal degrees of freedom.

Three dimensions are special in many ways. In particular, this is the only number of dimensions in which non trivial knots can exist. Indeed, any attempt at tying a knot in higher dimensions will fail because the strands can simply be moved into the extra dimensions to unravel the knot. In fewer dimensions, the opposite problem arises, the strands cannot be moved across each other to form a knot. It is exactly this property of three dimensions that give rise to anyons. In 2+1-dimensional spacetime the world-lines of particles can form non-trivial braids, giving rise to particle statistics known as braid statistics.

% In this thesis we will primarily be concerned with non-abelian anyons.
Particle statistics gives rise to an effective repulsion, known as statistical repulsion. This is well known in the case of bosons and fermions. Anyonic statistics is a topic of current research, focusing mainly on the case of abelian anyons. The statistics of non-abelian anyons is largely unexplored.

In this thesis we extend some the results for abelian anyonic statistics to the case of non-abelian anyons. Furthermore, we show how topological quantum computation (TQC) can be performed with non-abelian anyons. TQC provides topologically stable quantum states, in principle solving the issue of decoherence that exists with `ordinary' quantum computers.

As we build up the theory of anyons, many topics of mathematics are touched upon. Topology and representation theory is used to characterize braid statistics. The study of statistical repulsion uses tools from analysis and differential geometry. The abstract model for anyons uses results from category theory and theories rooted in quantum field theory.
